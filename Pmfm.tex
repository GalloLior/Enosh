\documentclass[14pt,oneside]{amsart}

\setlength\topmargin{-0.5in}
\setlength\headheight{15pt}
\setlength\headsep{15pt}
\setlength\footskip{25pt}
\setlength\textwidth{6.5in}
\setlength\oddsidemargin{0in}
\setlength\evensidemargin{0in}
\setlength\parindent{0.25in}
\setlength\parskip{0pt} % was 3pt
\setlength\textheight{9.0in}

\usepackage{amssymb,latexsym,amsmath}
\usepackage{graphicx,xcolor}
\usepackage[round]{natbib}
\usepackage{booktabs}
\usepackage{multirow}
\usepackage{pdflscape}
\usepackage{datetime}
\usepackage{afterpage}
\usepackage{graphicx}
\usepackage{dcolumn}
\usepackage{lscape}
\usepackage{pdfpages}
\usepackage[justification=centering]{caption}
\usepackage[toc,page]{appendix}
\setcounter{MaxMatrixCols}{10}
\providecommand{\jel}[1]{{{JEL classification:}} #1}
\providecommand{\keywords}[1]{{{Key words:}} #1}


\title{Post Modernism: Structural, Empirical Analysis.}
\author{Lior Gallo \\
The Bank of Israel \\
The Hebrew University of Jerusalem \\
This Version: \today }

\begin{document}
\maketitle
	\begin{abstract}

	\end{abstract}


\tableofcontents

\newpage

\section{Introduction}
I am going to go and build my first AI mechine. I am going to call it ENOSH. Bey.


The most magnificant thing about this thing is that it is real. As a matter effect the very existance of this thing means that there is no such thing as real. They call it postmodernisem. It is a story we tell ourselfe. It says that everything is subjective. There is no objective truth. I just thought about an example. I am currenlty setting in a bar looking at me pc. The pc's welcome screen presents spactacular picture of the grand canion. At the top left corner of the screen there are words askking me if I think it is a dramatic picture. I like it. I like the fact that the word perfectlly definces what have just happend. The level of fittnes of the word to what happens in the picture acttully ticcled the good side of my ADHD. This is twic that my biology reacted to some external stimuluse. First the actual drama in the picture which made me strar on it. Second was the time where the word fited the situation. 

Lets talk about the word drama. The word conveis so much. Lets start with the definition of this word:
Google\footnote{25.12.18, 20:49}: "Drama is the specific mode of fiction represented in performance: a play, opera, mime, ballet, etc, performed in a theatre, or on radio or television."

So it is a "mode of fiction". The could have added picture. Pictures causes drama. Pictures causes mode of fiction.

Let us compare it to what happend to me. I read the word an felt somthing. The word is obviuslly a simbule. A simbule we gave to some feeling. The feeling is not drama - it is thrill or exitment. My bloode presure changed. Some nirological system reacted and I felt something in my brain. And I tagged it. Drama. A collection of feelings which was caused by exosuer to a picture that presents a conflict of forces. Is it true. An objective trure. Were we not thought to be impressed by drama. Or is drama was there all the time a natural combination of feelings and we taged it with a name. Is there an objective truth or is it all subjective. 

I actually like the merriam-webster definition of: "a state, situation, or series of events involving interesting or intense conflict of forces"\footnote{https://www.merriam-webster.com/dictionary/drama}. 
This is it. The picture illustrate the power of nature. We people call it drama. Drama is an institution. As a matter effect every word is an institution. Here I am searching around for institution and they all around me. Every word I read write, every product around, every gesture, the floor on which i am standing, the fact I need to go home, my home, every thing. Sheet, it feels like a glance to the matrix. Ok, so we need to make some order. That is it. Ever since I know myself I feel as if most of the people understand the world better then me. Sodenlly it all makes scence. It is truth. They do understand the social world much better then me. This is not to say that they have some suoperiorety over me or me over them. I am simply differant. What is considerd as a disadvantage is on some situations a huge advantage. But society dose not like it. People like me were considernd strange, odds or symply differant. Society tends to surpress us. I belive that at some point they used to lock people like me or even cut some part of their brain. Well i am obviuslly agsagerating. But there were people whos brain was cutten just because some people classified them as different. 
 


Social buisness cicels - structure causes NIVUN becase it depress the differant. OK let us build an possible econometric indexes for institutions. 
1. Words. like motifes of stalus michaelopolis. Close. Take the dictionery. Build the words network. Build the well known indices of networks. Build this network along the years. OK you are stuck. English is a new langughe. What are you doing backward when there was only latin for example. 
2. Can you identify through the years "How dose society treated the differant". Take an example. A society is born with subjective subjects. As time goes by they build institutions. As institutions get older they become stronger and more depresing. A revoltion happens when society reach a point where it cannot hold itselfe. Actuallty this part is not compleate. Maybe it is a point where loses are more then benefits. But then it is simply efficiant institutions. You can construct a two sided over shooting model for this and if you have an index you can actually test it. 

Try to explain the following social institutions:
1. Gender privileges.
2. Rationalizem over imotionalizem privileges.
3. Eye contact over touch.
4. speaking more then writing
5. Writing more than painting.

My understanding of Derrida is :"The only way to deconstruct is thorough construction." 

The algebraic model should include endogenous construction of institutions as well as an endogenous mechanism of de-construction. The model will build tension between stability and progress. The construction mechanism will have a decreasing return. It will also have an expansion mechanism. As the world becomes more constructed the added value of to growth decreases. Deconstruction and dis-enthroling from the dogmas of the past takes the progress back however increases its marginal contribution to expansion. Stability is achieved when there is balance between the two. 


What are institutions?
Institutions are a social contract that benefits some group or other contract over the others. The benefit can be economical, social or mental. The most prevalent example of institution is war. There is an an-written social contract around the world that men should die in war rather then women. Women according to the jewish law are not concision enough to serve as a witness in court. Women were considered property for some time in history. 

However the examples above take this concept a step further. Not just people are discriminated in terms of privileges but also concepts, stories and methods or simply contracts. Contracts are privileged over others and so on. How do we formulate stories? 

entering a coalition.
adding people to the coalition.
Now add contraction to the story.
The productivity of contracts changes with time - the distribution changes.
The distructive force is a force which breaks previuse contracts.


Possibilities to formulate story:
A story is something the means by which subjective preferences are being suppressed. It is what causes the gap between individual's preferances and equilibrium preferances. A person is born with subjective prefeances 
 
There are stories through which stories are transfered: Languish, painting etc are stories. Or maybe not. Meybe we will call it technology.

Who do new 


Who do I express these 


Micoeconomics adjustment: Individuals has continuances preferences that are spanned on an interval of a size of one. A new product is a new point on the interval ($x$) and the preferences are some density function of this product $f(x)$. You start here and go a long way with microeconomics.

\bigskip

In this paper I develop and estimate a test for the coherence between the theory and the empiric of the post modern movement. My source of information on the theory of post-modernism is Wikipedia. For the analytic framework I formulate model which elevate the discussion. The algebraic model in this paper provides an formal representation of post-modernism from the fundamentals that dictate its structure, to the conduct and the performances of people that follow it. To test for the coherence of the theory with data I use the world value survey. This survey provides information on the verity of question regarding moral and politics. The survey has been conducted in several countries and for several years which allows us to use both cross section and time series methods. 
%%%%%%%%%%%%%%%%%%%%%%%%%%%%%%% Appendices %%%%%%%%%%%%%%%%%%%%%%%%%%%%%%%%%%%%

\bigskip

The attempte to classify science on the spectrume of truth reflects most of all missunderstantding of either the concept of science or the scale. Sience is agnostic to the truth. Einstein darwin and others claimed that they were agnostic and hence indipendent to the question of truth. On the graphical level we would say that the vertic on which science layes is orthogonal to the vertic of truth. Hence, They can describe a X-Y surface.   
\bigskip

Stracturalizem vs phenomenology
Rationalizem vs Empiricesem
 

\bigskip



"Hannah Arendt's The Origins of Totalitarianism, isn't a fundamental feature of the rise of the Nazi regime the disrespect for distinction between truth and lies?" \footnote{https://www.philosophytalk.org/blog/postmodernism-blame-post-truth}


What is science :

My attempts:

1. The mean by which humanity transfer information.
	Information is revealed by empirical methods.
	Empirics cannot confirm a theory.
	Empirics can only reject assumed hypothesis.
	The languish of science is mathematics.
	Rejection of theories is done by the applied mathematics field of statistics.

2. The mean by which humanity gather information.
	As if it is an instrument of revealing the truth.
	Is there truth? 
	
3. The mean by which humanity gather information.


\bigskip


Formal Definitions of science:

\bigskip

Google\footnote{19.12.18, 21:33}: " the intellectual and practical activity encompassing the systematic study of the structure and behavior of the physical and natural world through observation and experiment."

\bigskip

Wikipedia\footnote{19.12.18, 21:33}: " Science (from the Latin word scientia, meaning "knowledge")[1] is a systematic enterprise that builds and organizes knowledge in the form of testable explanations and predictions about the universe."

\bigskip

\textit{"Occam's razor (or Ockham's razor) is a principle from philosophy. Suppose there exist two explanations for an occurrence. In this case the one that requires the least speculation is usually better. Another way of saying it is that the more assumptions you have to make, the more unlikely an explanation."}

\bigskip

Formal definition of mathematics:

Google\footnote{19.12.18, 10:10}: "the abstract science of number, quantity, and space. Mathematics may be studied in its own right ( pure mathematics ), or as it is applied to other disciplines such as physics and engineering ( applied mathematics )."

\bigskip

Harari Ted:
I can have many identities - i can be loyal to many groups
The conflicts of different groups a problem.
But who ever said that life is easy? Stoic
Fascism is what happens when people ignore the complications
How dose fascist decides about the quality of a movie - dose it serves the nation: the truth dose not matter at all - there is no truth - Fascism is Post modernism.

Post-modernism is a strand of thought which is build on the concept that there is no true. 

The mainstream philosophers discussed two approaches for truth: Rationalizem (Decart) suggests that all truth is subjective. There is no objective truth. On the other extreme David Heoum and John look were empiricist. We know nothing but the things we observe. To reconcile between these approaches \cite{kant} presented an idea which draws on the ideas of Bayes's which later was developed to Bayesian analysis in statistics. This concept suggests that there is some prior (rational) knowledge which is updated with posterior (emirical) observation. According to \cite{Kant} there is some objective reality and we update our knowledge as we observe reality. This line of thought led to the development of Epistemology on one hand the existentialist (Nitshe Sarter and Kami) on the other.

Truth - objective or subjective?
Judgement - free or oppressed.
Kant - free will is a must. otherwise none can claim that anyone is responsible to his actions.

What will happend if something will attack science? Should science attack back? 
The importance of the human knowledge boundary :  how many time can I change? 
First line reality - objective or subjective?
Second line - reality or existence - same.
Third line politics - liberalism structuralism where i define it as how much of your junjment you give to others. 

I think there for I am - Decart - which is a good point.
 
$5+3=8$

$\sum_{MIKI1}^{MIKI2}$


$\prod$

\section{Empirical Analysis}
 
\begin{itemize}
\item Q "In political matters, people talk of "the left" and "the right." How would you place your views on this scale, generally speaking? (Code one number):"
\item Q "I am going to name a number of organizations. For each one, could you tell me how much confidence you have in them: is it a great deal of confidence, quite a lot of confidence, not very much confidence or none at all? (Read out and code one answer for each):"
\end{itemize} 

\section{Analytical Framework}
Michael Go Home


A subject maximize utility under constrained. Utility is defined as free choice on a continues interval between zero and one. A subject is constrained only by the decision of others. You can form a political party by cooperating with someone who share your own preferences you can force your choice on others. This is unless they cooperate on larger scales and then they force their choice on you. At this point we will remain in framework of cooperation on a single decision. If preferences are uniformly distributed then the subjects happiness will also be uniformly distributed and the society will be completely equal. 

\bigskip


\newpage
\setcounter{table}{0} \renewcommand{\thetable}{B\arabic{table}}
\begin{appendices}


\section{Data Appendix}

In order to conduct this analysis, we collected data and created a balanced panel dataset for the following countries(38): Australia, Austria, Belgium, Canada, Chile, Denmark, Estonia, Finland, France, Greece, Hungary, Iceland, Ireland, Israel, Italy, Japan, Luxembourg, Mexico, Netherlands, New Zealand, Norway, Poland, Portugal, Slovakia, Spain, Sweden, Switzerland, Turkey, United Kingdom, United States, Germany, Slovenia, Bulgaria, Croatia, Cyprus, Latvia, Lithuania and Romania.

For each of the following countries we collected data on six branches of economy: Agriculture, Mining and Querying, Manufacturing, Commercial
Services and Construction. In addition to consumption by Households and Gross Consumption.

A list of the data collected:

\begin{enumerate}
  \item Electricity consumption and its breakdown by sectors in million Kilowatts per hours, as published by the UNSTATS. Covering 1990-2016.
  \item GDP and its breakdown at constant 2010 prices in US Dollars, as published by the UNSTATS National accounts main aggregates database. Covering 1970-2016.
  \item Per capita GDP at current prices – US dollars as published by the UN statistics division. Covering 1970-2016.
  \item Access to electricity (\% of population), published by the World Bank. Covering 1990-2014.
  \item Capital stock at current PPPs (in mil. 2011US\$), published at Penn World Table version 9.0. Covering 1950-2014.
  \item Urban population (\% of total), published by the World Bank. Covering 1960-2016.
  \item Mean monthly and annual Country precipitation and country temperature for the period 1961-1999. Published by the World Bank.
  \item Average annual wages, In 2017 constant prices at 2017 USD PPPs as published by The OECD. Covering 1990-2017.
\end{enumerate}

\subsection{Sources Limitations}
		our dataset is limited in several aspects:
		 \begin{enumerate}
  \item UNSTATS breakdown to sectors of electricity consumption was not paralleled to UNSTATS breakdown to sectors of states GDP (ISIC Rev. 4). therefore we used the UNSTATS "guidelines for the 2016 united nations statistics division annual questionnaire on energy statistics" in order to correspond to ISIC Rev. 4.
  \item In the World Bank's Access to electricity dataset, All states covered in this paper has reached 100\% already at the year of 1990 with no changes throughout the years. 
  \item in order to have a balanced panel dataset for each sector, we had to make sure we have continuous series for each country in each sector. due to this reason we had to remove from our dataset the following states in the following sectors: 
  	\begin{enumerate} 
  	\item In the construction industry we removed the following counties:
		\begin{enumerate}   	 
  	  	\item Slovenia, due to lack of data for 1997-1999
  	  	\item Bulgaria, due to lack of data for 1994-2004
  	  	\item Greece, due to lack of data for 2015
  	  	\item Israel, due to lack of data for 2013-2015
 	  	\item Germany, due to lack of data for 2003-2015
 	  	\item Luxembourg, due to lack of data for 1990-1999
 	  	\item Slovakia, due to lack of data for 1992
 	  	\item Cyprus, due to lack of data for 1990-2006
 	  	\item Latvia, due to lack of data for 1990-2006
 	  	\item Lithuania, due to lack of data for 1990-2006
 	  	\item Malta, due to lack of data for 1990-2009
 	  	\item United States, due to lack of data for 1990-2002
 	  	\item Chile, No data at all
 	  	\item Republic of Korea, No data at all 	  	
 	  	\item Canada, No data at all
 		\end{enumerate}
	\item In the Commercial Services industries we removed the following counties:
		\begin{enumerate}   	 
  	  	\item Lithuania, due to lack of data for 1990-2006
  	  	\item Bulgaria, due to lack of data for 1994-2004
  	  	\item Latvia, due to lack of data for 1998-2006
  	  	\item Malta, due to lack of data for 1998-2000
  		\end{enumerate}
  		\item In the Agriculture industries we removed the following counties:
		\begin{enumerate}   	 
  	  	\item Belgium, due to lack of data for 1990-1996
  	  	\item Malta, due to lack of data for 1990-2006
  	  	\item Germany, no data at all
  	  	\item Slovenia, no data at all
  	  	\item United States, due to lack of data for 1990-2001
  		\end{enumerate}
  		\item In the Mining industries we removed the following counties:
		\begin{enumerate}   	 
  	  	\item Luxembourg, due to lack of data for 1990-1999
  	  	\item Switzerland, no data at all
  	  	\item United Kingdom, due to lack of data for 1990-2009
  	  	\item Slovakia, due to lack of data for 1990-1994
  	  	\item Slovenia, due to lack of data for 1990-1996
  	  	\item Sweden, missing data for 2014
  	  	\item Bulgaria, due to lack of data for 1990-2004
 	  	\item Cyprus, due to lack of data for 1990-2006
 	  	\item Latvia, due to lack of data for 1990-2006
 	  	\item Lithuania, due to lack of data for 1990-2006
 	  	\item Malta, due to lack of data for 1990-2009
  		\end{enumerate}
  	\item Iceland was removed from both the Manufacturing industries and the Total groups, due to dramatic chances in the Icelandic economy, further discussion on the Icelandic case will be follow.
  	\end{enumerate}
\end{enumerate}

\newpage




\newpage
\bibliographystyle{plainnat}
\bibliography{OECD}

\newpage
\setcounter{table}{0} \renewcommand{\thetable}{B\arabic{table}}

	\end{appendices}
 \end{document}